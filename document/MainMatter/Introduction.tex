\chapter*{Introducción}\label{chapter:introduction}
\addcontentsline{toc}{chapter}{Introducción}

%El t\'ermino “credenciales” engloba las nociones de diplomas, t\'itulos acad\'emicos, calificaciones, autorizaciones o t\'itulos profesionales [\cite{1}]. Las credenciales tienen gran importancia dado que forman parte de la vida cotidiana: carnet de conducir, t\'itulo universitario, expediente acad\'emico, pasaporte, por mencionar algunas. Cada credencial ayuda a identificar o verificar la identidad, calificaciones u otra informaci\'on fundamental asociada a una persona.

%Concluir con \'exito una carrera universitaria ofrece la posibilidad de obtener un t\'itulo acad\'emico. El t\'itulo o diploma, es un tipo de credencial, un certificado que proporciona garant\'ia y valor a los conocimientos adquiridos durante los estudios. Con \'el se demuestra que se form\'o parte de un riguroso proceso de formaci\'on con las respectivas evaluaciones. Esa preparaci\'on est\'a respaldada por un conjunto acad\'emico y una instituci\'on con capital intelectual y laboral. Al ser un certificado acad\'emico, el t\'itulo universitario faculta para trabajar en el sector de inter\'es o en aquellas \'areas en las que sea un requisito imprescindible estar titulado. Generalmente, cuando se realizan estudios universitarios, la instituci\'on encargada de emitir el t\'itulo acad\'emico es la universidad en la que se estudia.

Concluir con éxito una carrera universitaria permite recibir un título académico. Este título es un tipo de credencial [\cite{1}], un certificado que proporciona garantía y valor sobre los conocimientos adquiridos durante la carrera. Con él se certifica que se formó parte de un proceso de formación, con la calidad y rigor de la institución que lo otorga. Esta titulación está respaldada entonces por dicha institución. El título universitario faculta para trabajar en el sector de interés o en aquellas áreas en las que sea un requisito imprescindible estar titulado.

%A pesar de los avances tecnol\'ogicos que existen, en Cuba, la forma en que se emiten y gestionan las credenciales acad\'emicas, a\'un no ha aprovechado las posibilidades y beneficios de la tecnolog\'ia digital.

A pesar de los avances tecnológicos que existen, la forma en que se emiten y gestionan las credenciales académicas en nuestro país aún no ha aprovechado las posibilidades y beneficios de la tecnología digital.

Aplicar las tecnologías actuales digitales a estos procesos aporta beneficios como [\cite{3,4}]: 
\begin{itemize}
\item Aumentar la eficiencia del intercambio y comprobación de credenciales.
\item Proporcionar formas más confiables y seguras de proteger y verificar las credenciales, lo que reduce la existencia de errores y la oportunidad de fraude.
\item Hacer más transparente y accesible el acceso a dicha información a los propios estudiantes y a los autorizados a ello pudiendo seguir la traza de sus evaluaciones y otras certificaciones obtenidas.
\end{itemize}

Diplomas, habilidades o transcripciones para el mundo educativo son ejemplos de credenciales que pueden hacerse digitales [\cite{68}]. Estas contienen información detallada que incluye a nombre de quién se emite, entidad emisora, logros alcanzados y otros criterios para expedirla.

%Las credenciales digitales son certificados digitales emitidos por una instituci\'on. Diplomas, habilidades o transcripciones para el mundo educativo son ejemplos de credenciales digitales [\cite{68}]. Estas contienen informaci\'on detallada que incluye a nombre de qui\'en se emite, entidad emisora, logros alcanzados y criterios para expedirla. Las credenciales digitales no son realmente diferentes de las credenciales f\'isicas, como, por ejemplo, un pasaporte. De igual forma, cuando se presenta el pasaporte en un control fronterizo, cuando se solicita un trabajo o un curso universitario, se debe poder demostrar que se poseen las credenciales necesarias.

Existe una brecha entre las habilidades requeridas por las instituciones y la garantía que pueden aportar a esto las transcripciones actuales [\cite{4,68,69}]. Las credenciales académicas tradicionales pueden no ser indicativas de la verdadera gama de habilidades que una persona ha aprendido y tiene para ofrecer. Esto puede ser frustrante tanto para los estudiantes como para las instituciones, así como para los empleadores que encuentran dificultades para verificar y evaluar a los posibles candidatos.

%La necesidad de credenciales digitales universalmente reconocidas y aceptadas es el resultado de la brecha entre las habilidades buscadas por las instituciones y la prueba de esas habilidades brindadas por las transcripciones actuales [\cite{4,68,69}]. Las credenciales acad\'emicas tradicionales pueden no ser indicativas de la verdadera gama de habilidades que una persona ha aprendido y tiene para ofrecer. Esto puede ser frustrante tanto para los estudiantes como para las instituciones, as\'i como para los empleadores que encuentran dificultades para verificar y evaluar a los posibles candidatos.

Debido a la importancia fundamental de la elección del candidato adecuado durante un proceso de contratación, el reclutador realizar una verificación rigurosa de las calificaciones de los candidatos y la legitimidad del contenido de las credenciales proporcionadas. Actualmente, los empleadores o reclutadores suelen comunicarse con el emisor de una credencial para verificar si el otorgamiento es legítimo y está vigente. Esta verificación está a cargo de la oficina de registros del colegio o universidad del solicitante, pero también pudiera estar a cargo de un intermediario externo [\cite{63,71}]. El proceso no solo lleva tiempo, puede ser molesto burocráticamente, y además llevar un costo adicional si pasa por un intermediario que cobra una tarifa por ello.

%Se producen m\'as problemas si el emisor cierra o los registros se pierden o son inaccesibles, lo que hace que la credencial sea casi imposible de verificar verdaderamente [\cite{4}]. revisar

%Debido a la importancia fundamental de la elecci\'on del candidato adecuado durante el proceso de contrataci\'on, el reclutador debe pasar por una verificaci\'on rigurosa de las calificaciones de los candidatos y la legitimidad del contenido de las credenciales proporcionadas. Actualmente, los empleadores o reclutadores deben comunicarse con el emisor de una credencial para verificar si el otorgamiento es correcto y est\'a actualizado. La verificaci\'on est\'a a cargo de la oficina de registros del colegio o universidad del solicitante, pero ocasionalmente est\'a a cargo de una empresa externa [\cite{63,71}]. El proceso no solo lleva mucho tiempo, sino que si la verificaci\'on pasa por un tercero, generalmente conlleva una tarifa. Se producen m\'as problemas si el emisor cierra o los registros se pierden o son inaccesibles, lo que hace que la credencial sea casi imposible de verificar verdaderamente [\cite{4}].

Disponer de credenciales digitales brindaría una mejor manera de compartir de forma segura y confiable las calificaciones obtenidas por una persona a lo largo de su vida. Esto no solo es ventajoso para los estudiantes y graduados, sino también para las instituciones educativas y los empleadores que necesitan información rápida, precisa y verificable sobre los solicitantes.

%Las credenciales digitales brindan una mejor manera de compartir las calificaciones obtenidas por una persona a lo largo de su vida. Esto no solo es ventajoso para los estudiantes y graduados, sino tambi\'en para las instituciones educativas y los empleadores que necesitan informaci\'on r\'apida, precisa y verificable sobre los solicitantes.

Con las credenciales digitales, existe un mayor nivel de coherencia, datos y confianza en el reconocimiento de habilidades y la toma de decisiones. Esto significa que las personas pueden mostrar sus habilidades de una manera viable y verificable [\cite{64}]. De esta forma los empleadores están bien posicionados para mejorar sus prácticas de contratación y las instituciones están mejor equipadas con vistas al crecimiento de sus programas.

%Con las credenciales digitales, existe un mayor nivel de coherencia, datos y confianza en el reconocimiento de habilidades y la toma de decisiones. Esto significa que las personas pueden mostrar sus habilidades de una manera port\'atil y verificable lo cual proporciona un contexto valioso [\cite{64}]. Adem\'as, los empleadores est\'an bien posicionados para mejorar sus pr\'acticas de contrataci\'on y las instituciones est\'an mejor equipadas para hacer crecer sus programas.

Uno de los desafíos importantes del mundo digital es que los formatos digitales son susceptibles a modificaciones; las credenciales falsas y las prácticas fraudulentas siguen aumentando cada vez más a medida que se expande la utilización de las tecnologías y las redes [\cite{65,70}]. Este tipo de fraude puede causar una serie de problemas para las empresas [\cite{2}]. Para puestos de trabajo de alta responsabilidad como un médico cirujano, el potencial de daño causado por alguien con credenciales falsas es considerablemente mayor. Además, se puede dañar la relación entre los proveedores de educación y las organizaciones de contratación, así como dañar la reputación del emisor de la credencial.

%Se requiere financiación para capacitación inesperada o para cubrir la rotación de empleados, la capacitación lleva más tiempo y los procesos internos se retrasan.

%Uno de los desaf\'ios importantes del mundo digital es que los formatos digitales son f\'aciles de modificar, las credenciales falsas y las pr\'acticas fraudulentas siguen aumentando cada vez m\'as a medida que se expande la utilizaci\'on de las tecnologías [\cite{65,70}]. Este tipo de fraude puede causar una serie de problemas para las empresas. Se requiere financiaci\'on para capacitaci\'on inesperada o para cubrir la rotaci\'on de empleados, la capacitaci\'on lleva m\'as tiempo y los procesos internos se retrasan. Para puestos de trabajo de alta responsabilidad como un m\'edico cirujano, el potencial de da\~no causado por alguien con credenciales falsos es considerablemente mayor. Adem\'as, se puede da\~nar la relaci\'on entre los proveedores de educaci\'on y las organizaciones de contrataci\'on, as\'i como da\~nar la reputaci\'on del emisor de la credencial [\cite{2}].

Una tecnología que ha demostrado ayudar a resolver problemas como la precisión de los datos, la seguridad de las transacciones y otros es la \textit{blockchain} [\cite{1, 66}]. Debido a sus características de descentralización, trazabilidad e inmutabilidad, esta tecnología es una solución adecuada para reducir el riesgo de falsificación de las credenciales digitales. En una \textit{blockchain}, las credenciales digitales se registran en una base de datos distribuida la cual se encuentra protegida criptográficamente [\cite{2}], lo que garantiza que las credenciales registradas sean difíciles de falsificar, modificar o eliminar, pero que siempre estén disponibles para su verificación o visualización.

%La \textit{blockchain}, una base de datos descentralizada y a prueba de falsificaciones [\cite{1, 66}], pudiera permitir la protección y el almacenamiento de estas credenciales digitales.

%La blockchain, una base de datos descentralizada, duradera y a prueba de falsificaciones [\cite{1, 66}], pudiera permitir la protecci\'on y el almacenamiento de estas credenciales digitales. Utilizando la tecnolog\'ia de blockchain, las credenciales digitales se registran en una ``base de datos distribuida siempre activa'' la cual se encuentra protegida criptogr\'aficamente, esto garantiza que las credenciales registradas sean dif\'iciles de falsificar, modificar o eliminar, pero siempre est\'en disponibles para su verificación o visualización [\cite{2}]. 

%Las credenciales digitales escritas en la blockchain tienen su información encriptada, lo que impide la visibilidad y la extracción de datos sin el consentimiento previo del propietario [\cite{4,67}]. Esto garantiza que, si bien la transacción registrada siempre está visible y se puede verificar fácilmente, no hay riesgo para la información del destinatario.

%Estas soluciones, además de permitir a los estudiantes acceder a sus registros 
%fácilmente, reducen tiempo y costos en los procesos de verificación y autenticación de las 
%instituciones. 
%
%Las principales características del Blockchain son la descentralización, trazabilidad 
%y la inmutabilidad. Estas implican que los datos registrados en blockchain son más 
%específicos, auténticos y seguros (Ezeudu et al., 2018). Esta tecnología puede resolver 
%los problemas de irregularidad de información y facilitar la confianza entre extraños por 
%sus características de descentralización e inmutabilidad. 
%En este contexto, la tecnología BC puede transformar el proceso de emisión de 
%certificados, creando datos enlazados que se encuentren encriptados, disminuyendo 
%así el fraude académico debido al que existiría un proceso de verificación confiable. 
%Las cadenas de bloques se pueden usar para almacenar hashes criptográficos (“huellas 
%digitales”) de los certificados, o para almacenar las propias reclamaciones. Por lo tanto, 
%una cadena de bloques puede asumir la función de un registro de certificado público .

%Una tecnología que ha demostrado ayudar a resolver problemas como la precisión de los datos, la seguridad de las transacciones y otros es la cadena de bloques. Blockchain es un sistema de registro que es transparente y se distribuye a cada nodo para que no se pueda negar la verdad de los datos; este sistema puede agregar seguridad y reducir el riesgo de corrupción de datos. Blockchain es la solución adecuada para reducir el riesgo de falsificación de certificados digitales, y proteger los datos de certificados digitales con un libro mayor que contiene datos de certificados digitales distribuidos a todos los nodos hará que los datos de certificados digitales sean más seguros y claros para todos.

%Una cadena de bloques es una base de datos descentralizada que utiliza marcas de tiempo para registrar datos cronológicamente. Una vez creados, los bloques de datos recién producidos son inalterables e irreversibles. Para dificultar la realización de fraudes, se utiliza el cifrado, que hace que los datos sean inalterables. . Debido a la capacidad de la cadena de bloques para retener datos basados ​​en el tiempo, es adecuada para archivar datos educativos como diplomas, material didáctico e informes. El uso de una cadena de bloques para mantener registros cronológicamente correctos del aprendizaje de los estudiantes, como el tiempo de instrucción, los archivos del curso y los puntajes de las pruebas, puede permitir el uso de datos individualizados para proporcionar a los estudiantes y educadores herramientas para monitorear, evaluar y compartir el progreso de los estudiantes. La adopción de un método de registro basado en criptografía garantiza la precisión de los datos y elimina la posibilidad de manipulación o eliminación de datos.

%Debido a la capacidad de la cadena de bloques para registrar los logros de los estudiantes de manera abierta y verificable, el camino de cada estudiante para obtener una educación sería más reconocido y apreciado. Esto da como resultado un mayor retorno de la inversión y menores gastos de hardware. El historial de aprendizaje de cada estudiante se mantiene y protege mediante la tecnología blockchain, lo que hace que los datos que contiene sean difíciles de modificar o eliminar. Debido a la integridad del software de cifrado, los administradores de red podrán distribuir y recuperar datos rápidamente.

%La tecnología Blockchain [5] es la columna vertebral de las criptomonedas modernas como Bitcoin, Ethereum. Es el mecanismo simple que proporciona sistemas de base de datos distribuidos de última generación con transparencia, disponibilidad para la recuperación de datos con seguridad y privacidad. Basados en modernas técnicas de consenso, los sistemas basados en blockchain tienen una resistencia absoluta a la modificación de datos. Esto es muy aplicable en los sistemas de gestión de certificados digitales para garantizar que los certificados sean a prueba de manipulaciones.

%Al proporcionar un almacenamiento de datos público, descentralizado y confiable, la cadena de bloques se ha convertido en una tecnología innovadora que atrae el interés de muchas áreas de aplicación.

A tono con los elementos expuestos, se identifica para este trabajo de diploma, la \textbf{problem\'atica} existente relacionada con la necesidad de las universidades cubanas de un servicio seguro, portable, escalable y sostenible para la gestión, emisión y validación de las credenciales académicas.

Como solución a la problemática existente se plantea crear un sistema de gestión de credenciales académicas basada en \textit{blockchain}. Esta solución estará conformada por dos partes: la lógica que describe el sistema con el manejo de los datos en la \textit{blockchain}; y la interfaz de usuario que garantiza un medio de interacción con el sistema. La lógica del sistema es el contenido del trabajo de diploma de José Alejandro Labourdette-Lartigue Soto para obtener el título de Licenciado en Ciencias de la Computación.

En base a lo planteado se identifica como \textbf{problema de investigaci\'on} del presente trabajo: ¿Se podrá crear una interfaz de usuario que garantice una experiencia agradable y segura a los usuarios del sistema de gestión de certificados académicos propuesto como solución? 

%¿Se podr\'a crear una plataforma de credenciales digitales basada en blockchain que d\'e acceso a un certificado a prueba de manipulaciones directamente verificable?\par

Para darle respuesta al problema de investigación se define como \textbf{objetivo general}: Implementar una interfaz de usuario para un sistema de gestión de credenciales académicas basada en \textit{blockchain}.

Con el fin de cumplir con el objetivo general, se definen los \textbf{objetivos espec\'ificos} que pretende el presente trabajo de diploma:
\begin{itemize}
\item Facilitar la interacción del usuario con el sistema.
\item Transmitir la información de manera precisa para evitar que el usuario cometa errores durante la interacción con la utilización de técnicas de validación de la información.
\item Ofrecer facilidad uso para una variedad de usuarios, desde principiantes hasta expertos.
\item Brindar una experiencia agradable, de modo que los usuarios estén subjetivamente satisfechos al utilizarla.
\item Realizar diferentes funciones del sistema de forma rápida y con esfuerzo mínimo.
\item Garantizar la seguridad del sistema al permitir el acceso a las funcionalidades del sistema según los permisos que posea el usuario.
%Que pueda ser de uso sencillo por tal y tal tipo de usuario
%Que no se pueda hacer tal cosa ….
%Que el tiempo de respuesta sea …
%Que sea robusta por …..
\end{itemize}
%Flexibilidad y eficiencia de uso. 
%
%\item Facilidad de aprendizaje: el sistema debe ser fácil de aprender para que el usuario pueda comenzar a trabajar rápidamente con el sistema.
%\item Eficiencia: El sistema debe ser eficiente de usar, de modo que una vez que el usuario haya aprendido el sistema, sea posible un alto nivel de productividad.
%
%\item Concisión: Es fácil hacer que la interfaz quede clara clarificando y etiquetando todo en exceso, pero esto lleva a que la interfaz se hinche, donde hay demasiadas cosas en la pantalla al mismo tiempo. Si hay demasiadas cosas en la pantalla, es difícil encontrar lo que está buscando y, por lo tanto, la interfaz se vuelve tediosa de usar. El verdadero desafío para hacer una gran interfaz es hacerla concisa y clara al mismo tiempo.
%\item Consistencia: Las interfaces coherentes permiten a los usuarios desarrollar patrones de uso. Los usuarios aprenderán cómo se ven los diferentes botones e íconos y los reconocerán y se darán cuenta de lo que hacen en diferentes contextos. Un diseño único con consistencia habla de un buen diseño de interfaz de usuario.
%\item Eficiencia: Un buen diseño de interfaz de usuario le permite realizar diferentes funciones de la aplicación de software o sitio web más rápido y con menos esfuerzo.


Con el propósito de cumplir con todos los objetivos planteados se definen las siguientes tareas:

\begin{itemize}
\item Asimilaci\'on de arquitectura y tecnolog\'ia establecida para el desarrollo de la interfaz.
\item Empleo de arquitectura y tecnolog\'ia establecida para el desarrollo de la interfaz.
\item Implementaci\'on de la interfaz de usuario.
\item Establecer comunicaci\'on con el backend \nomenclature[backend]{\textbf{Backend}}{La parte del sistema que no está contenida o cargada en la aplicación del usuario, generalmente la API, el servidor y las bases de datos} a trav\'es del servicio REST API. %\ref{sec:RESTApi}.
\item Validar soluci\'on propuesta.
\end{itemize}

La presente investigación está estructurada de la siguiente forma:

\subsubsection*{Capítulo 1: Estado del Arte}

El contenido de este capítulo está centrado en el estado del arte relacionado con el objeto de estudio. Se presenta un estudio realizado de las principales tecnologías y herramientas a utilizar. Se abordan diferentes conceptos de la interfaz de usuario, características fundamentales y los principios para el diseño de las mismas, así como algunos elementos de la arquitectura de la información para lograr un diseño usable y accesible.

%: Se aborda todo lo relacionado con la fundamentación teórica que sustenta la presente 
%investigación, acerca del estudio del estado del arte así como de las herramientas, lenguajes 
%En este capítulo se analizan aspectos teóricos que serán necesarios investigar para la correcta realización 
%del trabajo. 
%
%El contenido de este capítulo está centrado en el estado del arte relacionado con el objeto de estudio. Se 
%presenta un estudio realizado de las principales bibliotecas gráficas de desarrollo en sistemas GNU/Linux. 
%Se abordan diferentes conceptos de la interfaz gráfica de usuario (GUI), características fundamentales y 
%los principios para el diseño de las mismas, así como algunos elementos de la arquitectura de la 
%información para lograr un diseño accesible.
%
% Cuenta con un respaldo teórico de los conceptos 
%fundamentales para el entendimiento de la solución. Presenta el estudio realizado de las 
%tecnologías y herramientas para el desarrollo de la biblioteca de componentes de interfaz de 
%usuario.
%
%En este capítulo se aborda el estado del arte referente a la investigación y el marco de los 
%conocimientos acerca de las herramientas a utilizar.

\subsubsection*{Capítulo 2: Propuesta}
En este capítulo se profundiza en el problema a resolver a través de su descripción y se realiza una propuesta de solución. Se elige la metodología de desarrollo de \textit{software} más adecuada a nuestro proyecto y se organiza el trabajo acorde a ella. Se realiza una descripción detallada de la arquitectura, por la cual se rige el desarrollo haciendo énfasis en los puntos cruciales de esta. Se identifican problemas relevantes de la propuesta y se describe cómo serían solucionados.

%Se identifican los requisitos funcionales y no funcionales que deben tenerse en cuenta. Por último, a partir de los requisitos se realiza una descripción de los casos de uso y se presentan los diagramas relacionados con estos.

%Se identifican y describen los procesos del negocio específicamente aquellos que se van a 
%automatizar lo cual incluye el modelo de los procesos del negocio también las características del 
%sistema a través de los requisitos funcionales, no funcionales y la descripción de los casos de uso del 
%sistema.
%
%Se modelan y describen los diagramas que representan las funcionalidades del sistema, aplicando los 
%patrones de arquitectura y diseño seleccionados.
%
%corresponde a la presentación 
%de la estructura de la solución propuesta
 
%En este capítulo se profundiza en el problema a resolver a través de su descripción y se muestra el 
%modelo de dominio generado. Se realiza una propuesta de solución y se identifican los requisitos 
%funcionales y no funcionales que deben tenerse en cuenta. Por último, a partir de los requisitos se realiza 
%una descripción de los casos de uso y se presentan los diagramas relacionados con estos.

%Se realiza un análisis del dominio 
%actual para su compresión. Se presentan los requisitos funcionales y no funcionales de la solución 
%conjuntamente con sus descripciones, además de una vista general de la propuesta. Se muestran
%detalladamente los elementos utilizados en el proceso de desarrollo de los componentes de 
%interfaz de usuario de la biblioteca. Se realiza una descripción detallada de la arquitectura, por la 
%cual se rige el desarrollo haciendo énfasis en los puntos cruciales de esta.

\subsubsection*{Capítulo 3: Implementación y Pruebas}

El contenido de este último capítulo se enfoca en la implementación y validación de la interfaz propuesta. Se realizan las pruebas definidas para el sistema garantizando su correcto funcionamiento y culmina con la validación de la propuesta de solución.

%Muestra cómo va estar estructurada la implementación del sistema, contiene los 
%diagramas de componentes y de despliegue así como las pruebas correspondientes al sistema.

%se 
%encontrarán todas las pruebas realizadas en desarrollo y producción, donde se 
%medirán los tiempos de ejecución y métricas como la precision

%En este capítulo se detalla la propuesta de solución al problema planteado. Se describe la organización 
%del sistema en un diagrama de componentes y se especifican los estándares de codificación a utilizar. Se 
%realizan las pruebas definidas para el sistema garantizando su correcto funcionamiento y culmina con la 
%validación de la hipótesis de investigación.

%El contenido de este último capítulo se enfoca en la implementación y validación de la interfaz propuesta, 
%a través del Método Delphi, tomando como base los resultados que se presentan en los capítulos 
%anteriores
%
%Se analizan los componentes indispensables para el 
%desarrollo de la solución, diagramas de clases y las relaciones entre los distintos componentes que 
%conforman la aplicación. Se presenta el diagrama de componentes y las pruebas de software 
%aplicadas a la solución
%
% En este capítulo se muestran los diagrama de despliegue y de 
%componentes como elementos fundamentales a modelar en la implementación. Se describen los casos de 
%prueba para cada caso de uso
%
%En este capítulo se presenta la validación de los requisitos mediante técnicas y métricas. Se valida el 
%diseño de la solución y la implementación, a través de métricas y métodos de prueba de caja blanca y 
%caja negra específicamente haciendo uso de las técnicas de camino básico y partición equivalente 
%respectivamente. Se hace uso del método de caso de estudio para verificar el funcionamiento de la 
%biblioteca de componentes de interfaz gráfica. Además, se realiza la validación de las variables de 
%investigación para validar la solución