\chapter*{Introducción}\label{chapter:introduction}
\addcontentsline{toc}{chapter}{Introducción}

El t\'ermino “credenciales” engloba las nociones de diplomas, t\'itulos acad\'emicos, calificaciones, autorizaciones o t\'itulos profesionales [\cite{1}]. Las credenciales tienen gran importancia dado que forman parte de la vida cotidiana: carnet de conducir, t\'itulo universitario, expediente acad\'emico, pasaporte, por mencionar algunas. Cada credencial ayuda a identificar o verificar la identidad, calificaciones u otra informaci\'on fundamental asociada a una persona.\par

Concluir con \'exito una carrera universitaria ofrece la posibilidad de obtener un t\'itulo acad\'emico. El t\'itulo o diploma, es un tipo de credencial, un certificado que proporciona garant\'ia y valor a los conocimientos adquiridos durante los estudios. Con \'el se demuestra que se form\'o parte de un riguroso proceso de formaci\'on con las respectivas evaluaciones. Esa preparaci\'on est\'a respaldada por un conjunto acad\'emico y una instituci\'on con capital intelectual y laboral. Al ser un certificado acad\'emico, el t\'itulo universitario faculta para trabajar en el sector de inter\'es o en aquellas \'areas en las que sea un requisito imprescindible estar titulado. Generalmente, cuando se realizan estudios universitarios, la instituci\'on encargada de emitir el t\'itulo acad\'emico es la universidad en la que se estudia.\par

A pesar de los avances tecnol\'ogicos que existen, en Cuba, la forma en que se emiten y gestionan las credenciales acad\'emicas, a\'un no ha aprovechado las posibilidades y beneficios de la tecnolog\'ia digital. 

Agregar tecnolog\'ias digitales a estos procesos aporta beneficios como [\cite{3,4}]: 
\begin{itemize}
\item Aumentar la eficiencia del intercambio y evaluaci\'on de credenciales.
\item Proporcionar formas m\'as confiables de proteger y verificar las credenciales, lo que reduce la oportunidad de fraude.
\item Expandir el control de los alumnos sobre sus credenciales, permitiendo una historia verificable de aprendizaje a lo largo de toda la vida.
\end{itemize}

Las credenciales digitales son certificados digitales emitidos por una instituci\'on. Diplomas, habilidades o transcripciones para el mundo educativo son ejemplos de credenciales digitales [\cite{68}]. Estas contienen informaci\'on detallada que incluye a nombre de qui\'en se emite, entidad emisora, logros alcanzados y criterios para expedirla. Las credenciales digitales no son realmente diferentes de las credenciales f\'isicas, como, por ejemplo, un pasaporte. De igual forma, cuando se presenta el pasaporte en un control fronterizo, cuando se solicita un trabajo o un curso universitario, se debe poder demostrar que se poseen las credenciales necesarias.

La necesidad de credenciales digitales universalmente reconocidas y aceptadas es el resultado de la brecha entre las habilidades buscadas por las instituciones y la prueba de esas habilidades brindadas por las transcripciones actuales [\cite{4,68,69}]. Las credenciales acad\'emicas tradicionales pueden no ser indicativas de la verdadera gama de habilidades que una persona ha aprendido y tiene para ofrecer. Esto puede ser frustrante tanto para los estudiantes como para las instituciones, as\'i como para los empleadores que encuentran dificultades para verificar y evaluar a los posibles candidatos.

Debido a la importancia fundamental de la elecci\'on del candidato adecuado durante el proceso de contrataci\'on, el reclutador debe pasar por una verificaci\'on rigurosa de las calificaciones de los candidatos y la legitimidad del contenido de las credenciales proporcionadas. Actualmente, los empleadores o reclutadores deben comunicarse con el emisor de una credencial para verificar si el otorgamiento es correcto y est\'a actualizado. La verificaci\'on est\'a a cargo de la oficina de registros del colegio o universidad del solicitante, pero ocasionalmente est\'a a cargo de una empresa externa [\cite{63,71}]. El proceso no solo lleva mucho tiempo, sino que si la verificaci\'on pasa por un tercero, generalmente conlleva una tarifa. Se producen m\'as problemas si el emisor cierra o los registros se pierden o son inaccesibles, lo que hace que la credencial sea casi imposible de verificar verdaderamente [\cite{4}].

Las credenciales digitales brindan una mejor manera de compartir las calificaciones obtenidas por una persona a lo largo de su vida. Esto no solo es ventajoso para los estudiantes y graduados, sino tambi\'en para las instituciones educativas y los empleadores que necesitan informaci\'on r\'apida, precisa y verificable sobre los solicitantes.

Con las credenciales digitales, existe un mayor nivel de coherencia, datos y confianza en el reconocimiento de habilidades y la toma de decisiones. Esto significa que las personas pueden mostrar sus habilidades de una manera port\'atil y verificable lo cual proporciona un contexto valioso [\cite{64}]. Adem\'as, los empleadores est\'an bien posicionados para mejorar sus pr\'acticas de contrataci\'on y las instituciones est\'an mejor equipadas para hacer crecer sus programas.

Uno de los desaf\'ios importantes del mundo digital es que los formatos digitales son f\'aciles de modificar, las credenciales falsas y las pr\'acticas fraudulentas siguen aumentando cada vez m\'as a medida que se expande la utilizaci\'on de las tecnologías [\cite{65,70}]. Este tipo de fraude puede causar una serie de problemas para las empresas. Se requiere financiaci\'on para capacitaci\'on inesperada o para cubrir la rotaci\'on de empleados, la capacitaci\'on lleva m\'as tiempo y los procesos internos se retrasan. Para puestos de trabajo de alta responsabilidad como un m\'edico cirujano, el potencial de da\~no causado por alguien con credenciales falsos es considerablemente mayor. Adem\'as, se puede da\~nar la relaci\'on entre los proveedores de educaci\'on y las organizaciones de contrataci\'on, as\'i como da\~nar la reputaci\'on del emisor de la credencial [\cite{2}].

La blockchain, una base de datos descentralizada, duradera y a prueba de falsificaciones [\cite{1, 66}], pudiera permitir la protecci\'on y el almacenamiento de estas credenciales digitales. Utilizando la tecnolog\'ia de blockchain, las credenciales digitales se registran en una ``base de datos distribuida siempre activa'' la cual se encuentra protegida criptogr\'aficamente, esto garantiza que las credenciales registradas sean dif\'iciles de falsificar, modificar o eliminar, pero siempre est\'en disponibles para su verificación o visualización [\cite{2}]. 

Las credenciales digitales escritas en la blockchain tienen su información encriptada, lo que impide la visibilidad y la extracción de datos sin el consentimiento previo del propietario [\cite{4,67}]. Esto garantiza que, si bien la transacción registrada siempre está visible y se puede verificar fácilmente, no hay riesgo para la información del destinatario.

A tono con los elementos expuestos, se identifica para este trabajo de diploma, la \textbf{problem\'atica} existente relacionada con la necesidad de las universidades cubanas de un servicio seguro, portable, escalable y sostenible para la gestión, emisión y validación de las credenciales académicas.\par

En base a lo planteado se identifica como \textbf{problema de investigaci\'on}: ¿Se podr\'a crear una plataforma de credenciales digitales basada en blockchain que d\'e acceso a un certificado a prueba de manipulaciones directamente verificable?\par

Para darle respuesta al problema de investigaci\'on se define como \textbf{objetivo general}: Implementar una Interfaz de Usuario para una plataforma de credenciales digitales basada en blockchain.

Para cumplir con el objetivo general se definen los siguientes \textbf{objetivos espec\'ificos}:
\begin{itemize}
\item Asimilaci\'on de arquitectura y tecnolog\'ia establecida para el desarrollo de la interfaz.
\item Empleo de arquitectura y tecnolog\'ia establecida para el desarrollo de la interfaz.
\item Implementaci\'on de la interfaz de usuario.
\item Establecer comunicaci\'on con el backend a trav\'es del servicio REST API.
\item Validar soluci\'on propuesta.
\end{itemize}