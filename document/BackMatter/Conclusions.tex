\begin{conclusions}
%Al comienzo del proyecto se planteaba la creación de una web app basada 
%principalmente en la directiva, es decir, gestionando el club desde un rol más 
%administrativo. Después de realizar un análisis sobre la competencia, la cantidad de 
%aplicaciones enfocadas a la administración de un club era mayor y, además, estas 
%aplicaciones estaban mejor desarrolladas, dando lugar a un aumento de dificultad a la 
%hora de intentar hacernos un hueco en el mercado. Al contrario que las aplicaciones 
%enfocadas a los entrenadores, donde hacerse un hueco en el mercado es más fácil 
%gracias a que la mayoría de estas están obsoletas, como hemos observado en el análisis 
%de la competencia. Por este motivo, se decidió comenzar por este tipo de aplicaciones, 
%dejando para futuras versiones aumentar las características de la administración del 
%club dentro de la web app.
%Se ha desarrollado una aplicación web mediante un proceso de software, aplicando 
%lo aprendido durante el Grado en Ingeniería Informática, haciendo un seguimiento 
%mediante la metodología Scrum, la cual ha tenido que ser adaptada al equipo de 
%desarrollo debido a que solamente estaba formado por una persona. A medida que 
%avanzaba el proyecto, las estimaciones de tiempo de las tareas iban siendo cada vez 
%mejores, debido al conocimiento sobre la capacidad de trabajo del equipo. También se 
%ha actualizado continuamente el backlog, priorizando siempre las tareas negociadas con 
%el tutor.
%Una práctica mencionada pero no aprendida en la carrera ha sido Test Driven 
%Development (TDD) la cual me ha permitido mejorar como desarrollador a la hora de 
%pensar y crear pruebas. Esta práctica ha costado mucho esfuerzo ponerla en uso, 
%debido a que siempre hemos creado las pruebas una vez desarrollado el código.
%Una pequeña dificultad a la hora de crear el proyecto ha sido la capa de presentación, 
%debido al uso del framework Vue 3, con el cual el equipo de desarrollo no había trabajado 
%anteriormente. La curva de aprendizaje ha sido bastante más rápida en comparación a 
%sus competidores, sin embargo, este framework utiliza un tipo de programación reactiva 
%la cual al inicio es un poco compleja de comprender. 


%Este proyecto ha sido elaborado con la motivación clara de poder aunar toda la 
%información aprendida durante el grado de ingeniería informática, y usarla para mejorar el 
%funcionamiento del grupo scout Estrella Polar.
%Durante el desarrollo del trabajo, se ha realizado una tarea de investigación considerable, 
%puesto que la mayor parte de los contenidos relacionados con el proyecto no se incluyen en 
%las asignaturas vistas en el grado. Esto ha supuesto un gran ejercicio de persistencia en 
%recopilar la información adecuada, así como de diseñar e implementar la aplicación de 
%manera correcta, para evitar los errores trabajados durante los años de carrera.
%En cuanto a los conocimientos aprendidos durante el progreso del proyecto se encuentran 
%las nociones del panorama actual en los ámbitos de frontend y backend en aplicaciones 
%web, y, sobre todo, el hecho de poder afrontar un proyecto desde cero usando dichos 
%conocimientos. 
%La valoración de la aplicación desarrollada es muy positiva, puesto que se han conseguido 
%implementar los requisitos funcionales de manera correcta, y los usuarios externos que han 
%podido probar sus distintas funcionalidades han considerado que es una plataforma 
%verdaderamente beneficiosa para las necesidades del grupo scout.
%En definitiva, la creación de la aplicación web Phoenix, ha servido para mejorar y 
%favorecer la labor llevada a cabo por los grupos scout, un método de educación que sin 
%duda genera un impacto en la sociedad. Poder usar la tecnología como un instrumento para 
%promover la transmisión de valores esenciales hoy en día a la juventud, de manera 
%desinteresada, es sin duda una razón muy positiva por la que realizar este tipo de 
%proyectos.
\end{conclusions}
