\begin{conclusions}
%Al comienzo del proyecto se planteaba la creación de una web app basada 
%principalmente en la directiva, es decir, gestionando el club desde un rol más 
%administrativo. Después de realizar un análisis sobre la competencia, la cantidad de 
%aplicaciones enfocadas a la administración de un club era mayor y, además, estas 
%aplicaciones estaban mejor desarrolladas, dando lugar a un aumento de dificultad a la 
%hora de intentar hacernos un hueco en el mercado. Al contrario que las aplicaciones 
%enfocadas a los entrenadores, donde hacerse un hueco en el mercado es más fácil 
%gracias a que la mayoría de estas están obsoletas, como hemos observado en el análisis 
%de la competencia. Por este motivo, se decidió comenzar por este tipo de aplicaciones, 
%dejando para futuras versiones aumentar las características de la administración del 
%club dentro de la web app.
%Se ha desarrollado una aplicación web mediante un proceso de software, aplicando 
%lo aprendido durante el Grado en Ingeniería Informática, haciendo un seguimiento 
%mediante la metodología Scrum, la cual ha tenido que ser adaptada al equipo de 
%desarrollo debido a que solamente estaba formado por una persona. A medida que 
%avanzaba el proyecto, las estimaciones de tiempo de las tareas iban siendo cada vez 
%mejores, debido al conocimiento sobre la capacidad de trabajo del equipo. También se 
%ha actualizado continuamente el backlog, priorizando siempre las tareas negociadas con 
%el tutor.
%Una práctica mencionada pero no aprendida en la carrera ha sido Test Driven 
%Development (TDD) la cual me ha permitido mejorar como desarrollador a la hora de 
%pensar y crear pruebas. Esta práctica ha costado mucho esfuerzo ponerla en uso, 
%debido a que siempre hemos creado las pruebas una vez desarrollado el código.
%Una pequeña dificultad a la hora de crear el proyecto ha sido la capa de presentación, 
%debido al uso del framework Vue 3, con el cual el equipo de desarrollo no había trabajado 
%anteriormente. La curva de aprendizaje ha sido bastante más rápida en comparación a 
%sus competidores, sin embargo, este framework utiliza un tipo de programación reactiva 
%la cual al inicio es un poco compleja de comprender. 


%Este proyecto ha sido elaborado con la motivación clara de poder aunar toda la 
%información aprendida durante el grado de ingeniería informática, y usarla para mejorar el 
%funcionamiento del grupo scout Estrella Polar.
%Durante el desarrollo del trabajo, se ha realizado una tarea de investigación considerable, 
%puesto que la mayor parte de los contenidos relacionados con el proyecto no se incluyen en 
%las asignaturas vistas en el grado. Esto ha supuesto un gran ejercicio de persistencia en 
%recopilar la información adecuada, así como de diseñar e implementar la aplicación de 
%manera correcta, para evitar los errores trabajados durante los años de carrera.
%En cuanto a los conocimientos aprendidos durante el progreso del proyecto se encuentran 
%las nociones del panorama actual en los ámbitos de frontend y backend en aplicaciones 
%web, y, sobre todo, el hecho de poder afrontar un proyecto desde cero usando dichos 
%conocimientos. 
%La valoración de la aplicación desarrollada es muy positiva, puesto que se han conseguido 
%implementar los requisitos funcionales de manera correcta, y los usuarios externos que han 
%podido probar sus distintas funcionalidades han considerado que es una plataforma 
%verdaderamente beneficiosa para las necesidades del grupo scout.
%En definitiva, la creación de la aplicación web Phoenix, ha servido para mejorar y 
%favorecer la labor llevada a cabo por los grupos scout, un método de educación que sin 
%duda genera un impacto en la sociedad. Poder usar la tecnología como un instrumento para 
%promover la transmisión de valores esenciales hoy en día a la juventud, de manera 
%desinteresada, es sin duda una razón muy positiva por la que realizar este tipo de 
%proyectos.


%Con la realización de este trabajo de diploma, y en cumplimiento del objetivo inicialmente planteado se 
%obtuvo una interfaz gráfica de usuario, para el software de visualización de imágenes médicas que se está 
%desarrollando en el proyecto VISMEDIC, que incluye dentro de sus requisitos las principales 
%funcionalidades de algunos de los software para la visualización de imágenes médicas existentes en el 
%mercado. Está completamente basada en la política de Open Source y la misma cumple con los principios 
%básicos para el diseño de las interfaces de usuario. 
%La interfaz será capaz de proporciona una respuesta visual a las acciones del usuario, brindando un 
%conjunto de funcionalidades que se utilizaran para visualizar, analizar y manipular imágenes DICOM.
%Además, la misma cuenta con mayor aceptación por parte de los expertos y los médicos especialistas, 
%que la utilizada anteriormente en el software VISMEDIC. Un factor importante lo constituye la naturaleza 
%multiplataforma de la biblioteca gráfica de desarrollo con la que se construyó la interfaz, permitiendo su 
%fácil migración a plataformas libres GNU\Linux, lo cual es un paso de progresivo hacia la emancipación
%tecnológica que requiere Cuba en estos momentos, por lo que constituye una solución que en un futuro 
%cercano puede tener un impacto positivo a nivel nacional tanto en la economía como en la sociedad

%En este trabajo se realizó el estudio de las herramientas y tecnologías de desarrollo que se utilizaron en la 
%confección de la solución. Finalmente se escogieron las que resultaron más adecuadas para el desarrollo 
%del sistema. Se desarrolló la aplicación Web sobre la plataforma J2EE, específicamente con las 
%tecnologías GWT y Spring frameworks, con los sub proyectos Spring Web Service y Spring Security.
%Se utilizó la metodología RUP, se concibió la arquitectura del sistema y se generaron los demás artefactos 
%pertenecientes al diseño. Durante el flujo de trabajo de implementación se realizaron los diagramas de 
%componentes y de despliegue, lo que facilitó la comprensión de la comunicación entre los componentes y 
%la distribución física de estos para la implementación del sistema. Además, se diseñaron casos de prueba 
%para evaluar y valorar la calidad del sistema, logrando así una revisión final de las especificaciones del 
%diseño y de la codificación.
%Como resultado de este trabajo se desarrolló una interfaz de usuario para administrar los servicios 
%disponibles en la plataforma COMCEL que cumple con los requisitos establecidos inicialmente. Por todo lo 
%anterior expuesto se concluye que los objetivos trazados fueron cumplidos.

%El estudio de la metodología, tecnologías y herramientas utilizadas para el desarrollo del proyecto, así 
%como las relativas al desarrollo de la biblioteca de componentes de interfaz de usuario UIToolsBox
%permitió profundizar en los principales elementos y los aspectos novedosos a tener en cuenta durante 
%el proceso de desarrollo.
% Con el análisis de soluciones semejantes disponibles se determinó que no existe ninguna que cumpla 
%con los requisitos establecidos para el desarrollo de las interfaces de usuario del proyecto.
% Se analizaron las necesidades del proyecto para el desarrollo de las interfaces de usuario, a partir de 
%las cuales se determinaron los requisitos a tener en cuenta para la implementación.
% Se definió una arquitectura basada en servicios que garantiza la flexibilidad de la ejecución de la 
%propuesta de solución, lográndose una independencia entre las capas definidas y extensibilidad
%necesaria en la propuesta.

%Una vez finalizado el proyecto, las conclusiones extraídas son las siguientes:
%- Se ha diseñado una aplicación escalable que podría servir de base para un 
%propósito más ambicioso en un futuro, debido a las tecnologías utilizadas.
%- El conjunto de estas tecnologías conforma un ecosistema cohesionado que facilita 
%mucho el desarrollo de funcionalidades que requieran de interacción entre los 
%distintos componentes. 
%- En la parte del almacenamiento de datos, la elección de MongoDB como base de 
%datos no relacional, dota a la aplicación de una gran escalabilidad, permitiendo 
%que pudieran gestionarse una gran cantidad de datos, Además, el formato de sus 
%datos permite que las transformaciones posteriores requeridas resultes casi 
%inmediatas. 
%- Spring como eje central de la aplicación, combina a la perfección tanto con 
%MongoDB como con Vue.js.
%- En cuanto a Vue.js, es una tecnología que se encuentra ahora mismo en auge, y 
%que brinda una gran cantidad de posibilidades tanto visuales como funcionales. 
%Estas últimas, aunque no han sido tratadas en profundidad en este proyecto,
%podrían aportar mucho más a la aplicación, incluso planteando ampliar las 
%funcionalidades apoyado por una conexión directa entre MongoBD y Vue.js, que 
%sería interesante en ciertos momentos. 
%- Por último, tanto los tiempos de procesamiento de la parte del servidor como la 
%fluidez de la interfaz gráfica proporcionada por Vue.js, dan como resultado una 
%experiencia por parte del usuario rápida y dinámica, que permite la obtención de 
%la información requerida de forma satisfactorio como se pretendía desde un primer 
%momento.

\end{conclusions}
