\begin{recomendations}
%Como continuación de la entrega de este trabajo, el proyecto puede continuar 
%avanzando con características pendientes de implementar, desde las más básicas hasta 
%las más ambiciosas. Es necesario añadir, de forma prioritaria, las características más 
%destacadas de la competencia, que serán añadidas al backlog. Una característica 
%diferenciadora, interesante, pero a la vez muy compleja sería el estudio de datos, con el 
%objetivo de crear una Inteligencia Artificial para evitar lesiones, deduciendo en qué 
%momento un jugador es probable que se lesione. Para llevar a cabo esto, es necesaria 
%la ampliación del equipo, siendo importante la contratación de un científico de datos. 
%Esta nueva característica nos permitirá negociar con clubes de mayor tamaño, quienes 
%están interesados en este tipo de características. 
%Por otro lado, para el cumplimiento de las propuestas realizadas en el análisis de la 
%competencia (Punto 3 de este documento) es necesaria la ampliación del equipo con un 
%nuevo diseñador de interfaces de usuario (UX UI) encargado de crear interfaces mucho 
%más profesionales, consiguiendo mejorar de manera notoria la experiencia del usuario.

%Una vez concluido el desarrollo del módulo de administración de la plataforma COMCEL, cumplidos los 
%objetivos trazados y teniendo en cuenta que es una primera versión del sistema, se recomienda:
%Refinar las funcionalidades ya implementadas.
%Realizar mejoras a la apariencia visual del sistema.
%Someter el sistema a pruebas de calidad.

%Como trabajos futuros, se proponen una serie de ampliaciones que acerquen más la 
%aplicación a un uso real, aprovechando sobre todo la escalabilidad de la cual se ha 
%intentado dotar al proyecto. Estas ampliaciones son las siguientes:
%- Ampliar el número de servicios disponibles en la aplicación, siguiendo el orden 
%que se ha iniciado para conseguir que la interfaz sea siempre dinámica y sencilla, 
%aunque su contenido sea mucho mayor.
%- Ampliación del número de datos y de documentos insertado en la base de datos,
%pudiendo así analizar el comportamiento real de la aplicación en situaciones que 
%se acerquen más a una simulación de una ciudad real. 
%- Integrar un sistema para insertar los datos obtenidos de manera dinámica y 
%masiva, utilizando técnicas de Big Data para aprovechar al máximo las 
%características de la aplicación. 
%- Añadir nuevas funcionalidades a la aplicación tales como el control de usuario o 
%la posibilidad de descargar los documentos en más formatos.
%- Integrar técnicas de Open Linked Data que permitan la vinculación de los 
%documentos aportando mucho más valor a la información obtenida.
%- Realizar un análisis que prepare el camino para poder implementar la aplicación, 
%junto con las mejoras mencionadas anteriormente, en una ciudad real con la 
%infraestructura necesaria para obtener los datos de los diferentes sistemas 
%integrados en esta
\end{recomendations}
