\begin{resumen}
Las credenciales académicas son documentos que dan fé de la finalización exitosa de cualquier prueba, examen o actúan como una validación de la habilidad de un individuo. Actualmente, el dominio de la gestión de credenciales académicas adolece de un gran consumo de tiempo, alto costo, dependencia de terceros y falta de transparencia. Una solución basada en blockchain intenta resolver estos puntos débiles al permitir que cualquier reclutador o empresa verifique las credenciales del usuario sin depender de ningún tercero centralizado. Por lo tanto, se propone la investigación, análisis y comparación de aplicaciones basadas en la tecnología blockchain en el sector educativo y, además, la realización de una interfaz de usuario de un sistema de gestión de credenciales académicas basado en blockchain siguiendo las fases de análisis, diseño, implementación y pruebas.

%El proyecto habla sobre los detalles de implementación de la interfaz de usuario creada para un sistema de gestión de credenciales académicas basado en blockchain.


%se basa en BlockCerts, un proyecto del MIT que actúa como un estándar abierto para las credenciales de blockchain. El proyecto habla sobre los detalles de implementación de la aplicación descentralizada creada para BlockCerts Wallet. Es un intento de aprovechar el poder de la tecnología blockchain como notario global para la verificación de registros digitales.

%Este proyecto de licenciatura trata sobre el diseño y desarrollo de una aplicación web independiente en nombre de Enoco AS, una empresa que ofrece soluciones para automatizar, analizar, visualizar y aumentar la eficiencia del uso de energía para los edificios de sus clientes. Esta aplicación está destinada a servir como un producto complementario a su producto existente, Eurora. Eurora es una herramienta compleja para analizar y controlar el uso de energía en sus edificios. La aplicación diseñada en este proyecto es una aplicación de tablero mucho más simple, que se utilizará principalmente para visualizar datos de sensores, por ejemplo, en una pantalla grande en vestíbulos u oficinas.

%El 
%presente trabajo expone un estudio de varias bibliotecas de interfaz gráfica, se muestra la selección de
%las herramientas, tecnologías y se brinda una descripción de la solución de software. Se exponen 
%además, las disciplinas de la metodología de desarrollo y se realizó la validación de la solución
%mediante técnicas y métricas

%Por lo tanto, se propone la investigación, análisis y comparación del frameworks front-ends Vue 3 y,
%además, la realización de un pequeño prototipo de aplicación web realizada con esta herramienta siguiendo
%las fases de análisis, diseño, implementación y pruebas

%Toach App es una aplicación web enfocada a la gestión deportiva y administrativa 
%para clubes de fútbol. Permite a la directiva del club coordinar los equipos y sus 
%entrenadores. A su vez, el entrenador será la figura principal de esta aplicación; 
%permitiéndole organizar e individualizar los entrenamientos según las necesidades del 
%equipo. Sin embargo, el receptor final de esta aplicación es el usuario, que en este caso 
%será el jugador del equipo, que tiene acceso a un calendario personalizado, videos 
%explicativos y más funcionalidades. Se intenta conseguir una mejora en el uso de la
%tecnología facilitando la accesibilidad a este tipo de aplicaciones, por tanto, se puede 
%adaptar a cualquier dispositivo móvil. 

\textbf{Palabras clave} - Credenciales académicas, Interfaz de Usuario, Verificación, Blockchain.
\end{resumen}

\begin{abstract}
Academic credentials are documents that attest to the successful completion of any test, exam, or act as a validation of an individual's ability. Currently, the academic credential management domain suffers from high time consumption, high cost, third-party dependency, and lack of transparency. A blockchain-based solution attempts to address these pain points by allowing any recruiter or company to verify user credentials without relying on any centralized third party. Therefore, the research, analysis and comparison of applications based on blockchain technology in the educational sector is proposed and, in addition, the realization of a user interface of a blockchain-based academic credential management system following the analysis phases , design, implementation and testing.

\textbf{Keywords} - Academic credentials, User Interface, Verification, Blockchain.
\end{abstract}