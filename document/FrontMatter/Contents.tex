\tableofcontents
\listoffigures
\listoftables
% \listofalgorithms
\lstlistoflistings

\nomenclature[CSS]{\textbf{CSS}}{Las hojas de estilo en cascada (en inglés \textit{Cascading Style Sheets}), CSS es un lenguaje usado para definir la presentación de un documento estructurado escrito en HTML.}
\nomenclature[HTML]{\textbf{HTML}}{Lenguaje de Marcado de Hipertexto (en inglés \textit{HyperText Markup Language}), es el lenguaje de marcado predominante para la construcción de páginas Web. Se usa para describir la estructura y el contenido en forma de texto, así como para complementar el texto con objetos tales como imágenes.}
\nomenclature[HTTP]{\textbf{HTTP}}{(\textit{HyperText Transfer Protocol}): es un protocolo de transferencia de hipertexto es el usado en 
cada transacción de la web.}
\nomenclature[CRUD]{\textbf{CRUD}}{Crear, leer, modificar y eliminar (en inglés \textit{ Create Read Update Delete}), las cuatro operaciones básicas en el almacenamiento persistente como bases de datos.}
\nomenclature[DOM]{\textbf{DOM}}{Modelo en Objetos para la representación de Documentos es esencialmente una interfaz de programación de aplicaciones que proporciona un conjunto estándar de objetos para representar documentos HTML.}
\nomenclature[JSON]{\textbf{JSON}}{\textit{JavaScript Object Notation}. Formato de texto sencillo para el intercambio de datos.}
\nomenclature[Endpoint]{\textbf{Endpoint/punto de acceso}}{Nodo de una comunicación en red. En este caso los puntos de acceso de este trabajo son las diferentes rutas ofrecidas por el servidor.}
\nomenclature[Token]{\textbf{Token}}{Es una cadena de caracteres que contiene información, cifrada por el servidor, de un usuario en concreto.}
